% !TEX TS-program = pdflatex
% !TEX encoding = UTF-8 Unicode

% This is a simple template for a LaTeX document using the "article" class.
% See "book", "report", "letter" for other types of document.

\documentclass[11pt]{article} % use larger type; default would be 10pt

\usepackage[utf8]{inputenc} % set input encoding (not needed with XeLaTeX)

%%% Examples of Article customizations
% These packages are optional, depending whether you want the features they provide.
% See the LaTeX Companion or other references for full information.

%%% PAGE DIMENSIONS
\usepackage{geometry} % to change the page dimensions
\geometry{a4paper} % or letterpaper (US) or a5paper or....
% \geometry{margin=2in} % for example, change the margins to 2 inches all round
% \geometry{landscape} % set up the page for landscape
%   read geometry.pdf for detailed page layout information

\usepackage{graphicx} % support the \includegraphics command and options

% \usepackage[parfill]{parskip} % Activate to begin paragraphs with an empty line rather than an indent

%%% PACKAGES
\usepackage{booktabs} % for much better looking tables
\usepackage{array} % for better arrays (eg matrices) in maths
\usepackage{paralist} % very flexible & customisable lists (eg. enumerate/itemize, etc.)
\usepackage{verbatim} % adds environment for commenting out blocks of text & for better verbatim
\usepackage{subfig} % make it possible to include more than one captioned figure/table in a single float
% These packages are all incorporated in the memoir class to one degree or another...

%%% HEADERS & FOOTERS
\usepackage{fancyhdr} % This should be set AFTER setting up the page geometry
\pagestyle{fancy} % options: empty , plain , fancy
\renewcommand{\headrulewidth}{0pt} % customise the layout...
\lhead{}\chead{}\rhead{}
\lfoot{}\cfoot{\thepage}\rfoot{}

%%% SECTION TITLE APPEARANCE
\usepackage{sectsty}
\allsectionsfont{\sffamily\mdseries\upshape} % (See the fntguide.pdf for font help)
% (This matches ConTeXt defaults)

%%% ToC (table of contents) APPEARANCE
\usepackage[nottoc,notlof,notlot]{tocbibind} % Put the bibliography in the ToC
\usepackage[titles,subfigure]{tocloft} % Alter the style of the Table of Contents
\renewcommand{\cftsecfont}{\rmfamily\mdseries\upshape}
\renewcommand{\cftsecpagefont}{\rmfamily\mdseries\upshape} % No bold!

%%% END Article customizations

%%% The "real" document content comes below...

\title{Procesamiento de archivos  XML  en Python}
\author{Juan Alvarado,Estefania Lozano \&  Henry Lasso}
%\date{} % Activate to display a given date or no date (if empty),
         % otherwise the current date is printed 

\begin{document}
\maketitle

\section{Introducción}
El propósito de nuestro proyecto es que  mediante un analizador de texto en este caso el parseo comprobar si un archivo xml esta bien formado ,comprobar si el parseo es válido y poder realizar consultas sobre la información contenida en el archivo .\\ \\
El lenguaje de programación que utilizaremos es Python que cuenta con estructuras de datos eficientes y de alto
nivel y un enfoque simple pero efectivo a la programación orientada a objetos.\\ \\
La elegante sintaxis de Python y su tipado dinámico, junto con su naturaleza interpretada, hacen de éste un lenguaje ideal para scripting y desarrollo rápido de aplicaciones en diversas áreas y sobre la mayoría de las plataformas\\ \\
Python permite escribir programas compactos y legibles. Los programas en Python son típicamente más cortos que sus
programas equivalentes en C, C++ o Java por varios motivos:
\begin{itemize}
\item los tipos de datos de alto nivel permiten expresar operaciones complejas en una sola instrucción
\item la agrupación de instrucciones se hace por sangría en vez de llaves de apertura y cierre
\item  no es necesario declarar variables ni argumentos.
\end{itemize}

\section{Alcance}
Consiste en extraer la información que contiene un archivo xml, analizar su contenido , representarlo en una estructura y realizar funciones que nos permitan consultar su contenido.

\section{Descripción}


\section{¿Qué se implementó y qué no se implementó? }
\subsection{Lo implementado} 

\section{Observaciones}
\begin{itemize}
\item  
\end{itemize}

\section{ Conclusiones}
\begin{itemize}
\item  
\end {itemize}




\end{document}
